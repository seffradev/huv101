\documentclass[12pt]{article}

\usepackage[a4paper, total={16cm, 22cm}]{geometry}
\usepackage{parskip}
\usepackage[swedish]{babel}
\usepackage{csquotes}
\usepackage[natbib=true, style=numeric, sorting=none]{biblatex}

\addbibresource{references.bib}

\title{Artificiell intelligens, en diskussion om miljövänlighet och potentiella
tillämpningar}
\author{Hampus Avekvist}

\begin{document}

\maketitle

2022 slog företaget OpenAI igenom med sin stora språkmodell (egenöversättning
från large language model, LLM) ChatGPT \cite{chatGpt}. Den agerar likt en
chattbot, dock med en större kunskapsbas än tidigare, och ett
konversationsorienterat kommunikationssätt, vilket gör tjänsten tillgänglig
för gemene person. Diverse tjänster under paraplyfrasen artificiell intelligens
(AI) har funnits i flera år (myntandet av datorintelligens attribueras till
Alan Turing redan år 1947 \cite{historyOfAI}). De har i allmänhet utvecklats
för särskilda användningsområden, men ChatGPT slog igenom bland de bredast
tillgängliga. Det finns även nischade AI-verktyg som för programmering
\cite{copilot, tabnine} och bildgenerering \cite{dall-e, midjourney}. Verktygen
lämpar sig väl till att assistera och förenkla i vardagen, vilket gjort
tjänsterna populära. Dock, som konsekvens av bred användning, vad är
koldioxidavtrycket från dessa tjänster? Hur kan de blir mer miljövänliga?

Bruket av AI-verktyg, trots energianvändningen för deras drift, kan anses
värdefullt eftersom kompetenskapitalet växer. Om man, som i svag hållbarhet
\cite[15]{gullikssonHolmgren}, ser kompetensen som en resurs som väger upp
för verktygens utsläpp, i sammanhang som förhöjd produktivitet där varje
människa kan åstadkomma mer på kortare tid, så innebär vidareutvecklingen
av tjänsterna enbart goda ting. I \cite{carbonFootprintOfChatGpt} framgår
det (innan uppdateringen 2023-03-20) att CO$_2$-ekvivalenterna från ChatGPT
estimerades till 8.4 ton per år, dock efter artikeluppdateringen kan (med
enkla beräkningar i samma antaganden) ekvivalenterna uppnå 15225 ton per år.
Dessa mätningar gäller enbart för ChatGPT så tidigare nämnda tjänster bör
tas i beaktning i heltäckande argument. Här används enbart ChatGPT för att
illustrera tillväxten av användningen av denna typ av tjänster. Vid träning
av modeller som ChatGPT-3 och ChatGPT-4 nämns det i
\cite{environmentalImpactOfChatGpt} att estimerat 700 000 liter färskvatten
använts för de datacenter som modellerna tränats i. Då modellerna växer och
mer data används för träning är det osannolikt att förbrukningen minskar,
såvida metoderna inte förändras.

Ur en stark hållbarhetssynpunkt \cite{gullikssonHolmgren} bör modellerna
tränas enbart från förnybara energikällor. Detta är möjligt, om än i en
mindre skala. Dock utförs arbeten för stora helt hållbara datacenter, exempelvis
Googles, som siktar på ``net-zero carbon'' 2030 \cite{googleNetZeroCarbon}.
Vattenförbrukningen, som tidigare nämnt är relevant i dessa sammanhang, för
deras tjänster adresseras i \cite{googleWaterStewardship} som också siktar på
2030. Det innebär däremot att Google, inte utan att nyttja lagerresurser
\cite[16]{gullikssonHolmgren}, är redo att bedriva AI-verktyg i deras datacenter
innan 2030.

Det som poängterats hittills är att AI-verktyg blir mer populära, dock med
konsekvensen att de släpper ut mer i samband med att de skalar upp. Det följer
inte principerna för stark hållbarhet \cite[15]{gullikssonHolmgren}. Däremot
kan det inom kontextet för svag hållbarhet \cite[15]{gullikssonHolmgren} väga
upp som tidigare nämnt i kompetenskapital. Verktygen kan särskilt väga upp för
sina utsläpp om de används för att hitta lösningar på klimatproblemen och kan
effektivisera exempelvis transporter och industrier. Om AI används för att
optimera transport- och industrisektorerna åtminstone en procent (följande
tidigare beräkning och jämför med tabell 6.2 \cite[265]{gullikssonHolmgren})
kommer AIs \textit{energy returned on energy invested} (EROEI) redan
kompenserats för.

Sammanfattningsvis, AI växer i popularitet och kan användas för att höja
produktiviteten. Det leder dock till höjda utsläpp som kan mitigeras via
miljövänliga datacenter, som Googles initiativtagande. Däremot, redan nu,
om dessa verktyg och tjänster även används för att optimera stora
utsläpps-sektorer som transport och industri, där deras egen förbrukning kan
vägas upp, så ser redskapen ut som något bra som bör stanna för en mer hållbar
värld.

\printbibliography

\end{document}
