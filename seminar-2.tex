\documentclass[12pt]{article}

\usepackage[a4paper, total={16cm, 22.5cm}]{geometry}
\usepackage{parskip}
\usepackage[swedish]{babel}
\usepackage{csquotes}
\usepackage[natbib=true, style=numeric, sorting=none]{biblatex}

\addbibresource{references.bib}

\title{Informationsteknik och vattenrening -- idéer om dataingenjörens del }
\author{Hampus Avekvist}

\begin{document}

\maketitle

Människans förbrukning av miljön är inte hållbar. Vi svenskar
har ett ekologiskt avtryck som motsvarar ungefär fyra jordklot
\cite[262-263]{gullikssonHolmgren}. Något vi har gott om än så
länge är färskt vatten, något som inte är en självklarhet för
resten av världen \cite[276]{gullikssonHolmgren}. Det finns
möjligheter att förbättra kvaliteten av vatten och åtkomsten
till de länder där det behövs, bland annat med datateknik och
allmän informationsteknik (IT). Därför ämnar denna text att
utforska vattenrening i ett sammanhang med IT och går sedan
djupare in i hållbarheten gällande IT som helhet.

\paragraph{Vattenrening}

Google har som mål att rena 120\% av vattnet de förbrukar för
driften av deras datahallar och kontor
\cite{googleWaterStewardship}. Det försöker de uppnå bland annat
genom att ta hänsyn till vad för vattenanvändande faciliteter de
installerar (exempelvis toaletter och kranar) som är uttänkt
effektivare. De ser även till att främst använda vatten som inte
går att använda i hygieniska situationer som matlagning och
rengöring, när de kyler sina hallar. 

Dock, från en dataingenjörs perspektiv, är det viktigaste Google
gör att arbeta datadrivet \cite{googleWaterStewardship}. Att
arbeta datadrivet innebär att beslut fattas från mätdata och inte
på grund av andra källor som trender och upplevelser. Ett
datadrivet arbetssätt går att anamma genom installation av
internet of things (IoT)-enheter som samlar relevant data att
bearbeta.

Exempel på data att samla är vattenförbrukning, vattenkvalité och
temperaturen i datahallar kontra temperaturen utomhus. Genom att
låta insamlad data processas av artificiell intelligens (AI) kan
förslag automatiskt föreslås och diskuteras för bättre
miljöanpassning. 

Dataingenjörer behövs i dessa sammanhang, både för att konstruera
IoT-sensorerna och nätverken, men även för att designa
AI-modellerna. Vidare kan beslut från AI-förslag tas automatiskt
där det krävs ett mjukvarukoppel också utvecklat av dessa
ingenjörer. Som \cite[282]{gullikssonHolmgren} säger, IT kan
nyttjas för automatisering, särskilt i dessa fall för ökad
hållbarhet.

\paragraph{Informationsteknik}

2008 stod IT för 2-2.5\% av världens totala koldioxidutsläpp
\cite[283]{gullikssonHolmgren}. Branschen har vuxit samtidigt som
tekniken har blivit effektivare, så estimat (inkluderande
underhållning och media, E\&M) hamnar på 2.6\% där 1.4\% är
för just information- och kommunikationsteknologi
\cite{ictFootprints}. Distinktionen framgår inte i
\cite[283]{gullikssonHolmgren} om det räknar med E\&M-sektorn så
fortsättningsvis exkluderas det även ur siffror från
\cite{ictFootprints}.

Utöver mål att bli ``kolfria'' genom alternativa energikällor,
som Googles \textit{net-zero carbon} \cite{googleNetZeroCarbon},
kan mjuk- och hårdvaruprocesser i sig själva effektiviseras. Det
finns exempelvis \textit{green software} som syftar till att göra
mjukvara miljövänligt \cite{greenSoftwareFoundation}. Den
generella tanken kring miljövänlig mjukvara är att göra mindre.
I en form ska kod vara optimerad så färre CPU-cykler och
därav mindre energi krävs för att utföra en uppgift. En annan
form, fortfarande i spåret att göra mindre, är att stänga av
maskiner (skala ner) när de inte behövs och starta igen (skala
upp) när de behövs. Då är datorer inte igång utan anledning.
En extremare variant är att enbart drifta tjänster när energin
maskinerna använder är från förnybara källor. Detta går att
automatisera med Carbon Aware SDK \cite{carbonAwareSdk} från
Green Software Foundation.

Hittills har enbart ekologiska aspekter för informationstekniken
diskuterats. Ämnet är större än så, där resten av detta arbete
kommer diskutera den sociala påverkan av IT och avstår det
ekonomiska för att illustrera det moraliska och etiska ansvar
dataingenjörer har. IT främjar informationsspridning men en fråga
ställs i \cite[284]{gullikssonHolmgren}: Vem får vara med? En
annan fråga att ställa är: Vad får vara med? Dessa frågor öppnar
för diskussion om censur och begränsningar, samt på vilka villkor
får man delta?

IT och särskilt internet öppnar möjligheten att få svar på nästan
vad som helst inom ett par sekunder. Däremot drivs de flesta
tjänster och system av privata aktörer som har makten att själva
välja vad som ska synas och inte. För att behålla ett öppet och
neutralt internet behöver utvecklarna för dessa tjänster ta ett
ställningstagande och stå för att hålla yttrandefriheten stark
och kämpa emot företagsbeslut att implementera censurerande
verktyg. Om en dataingenjör ska utföra ett hållbart arbete bör
socialt hållbara politiska åsikter genomsyra tjänster och
produkter som utvecklas, även om det kan leda till en uppsägning.

\printbibliography

\end{document}
